%%%%%%%%%%%%%%%%%
% This is an sample CV template created using altacv.cls
% (v1.3, 10 May 2020) written by LianTze Lim (liantze@gmail.com). Now compiles with pdfLaTeX, XeLaTeX and LuaLaTeX.
% This fork/modified version has been made by Nicolás Omar González Passerino (nicolas.passerino@gmail.com, 15 Oct 2020)
%
%% It may be distributed and/or modified under the
%% conditions of the LaTeX Project Public License, either version 1.3
%% of this license or (at your option) any later version.
%% The latest version of this license is in
%%    http://www.latex-project.org/lppl.txt
%% and version 1.3 or later is part of all distributions of LaTeX
%% version 2003/12/01 or later.
%%%%%%%%%%%%%%%%

%% If you need to pass whatever options to xcolor
\PassOptionsToPackage{dvipsnames}{xcolor}

%% If you are using \orcid or academicons
%% icons, make sure you have the academicons
%% option here, and compile with XeLaTeX
%% or LuaLaTeX.
% \documentclass[10pt,a4paper,academicons]{altacv}

%% Use the "normalphoto" option if you want a normal photo instead of cropped to a circle
% \documentclass[10pt,a4paper,normalphoto]{altacv}

\documentclass[10pt,a4paper,ragged2e,withhyper]{altacv}

%% AltaCV uses the fontawesome5 and academicons fonts
%% and packages.
%% See http://texdoc.net/pkg/fontawesome5 and http://texdoc.net/pkg/academicons for full list of symbols. You MUST compile with XeLaTeX or LuaLaTeX if you want to use academicons.

% Change the page layout if you need to
\geometry{left=1.2cm,right=1.2cm,top=1cm,bottom=1cm,columnsep=0.75cm}

% The paracol package lets you typeset columns of text in parallel
\usepackage{paracol}

% Change the font if you want to, depending on whether
% you're using pdflatex or xelatex/lualatex
\ifxetexorluatex
  % If using xelatex or lualatex:
  \setmainfont{Roboto Slab}
  \setsansfont{Lato}
  \renewcommand{\familydefault}{\sfdefault}
\else
  % If using pdflatex:
  \usepackage[rm]{roboto}
  \usepackage[defaultsans]{lato}
  % \usepackage{sourcesanspro}
  \renewcommand{\familydefault}{\sfdefault}
\fi

% ----- LIGHT MODE -----
\definecolor{SlateGrey}{HTML}{2E2E2E}
\definecolor{LightGrey}{HTML}{666666}
\definecolor{PrimaryColor}{HTML}{001F5A}
\definecolor{SecondaryColor}{HTML}{0039AC}
\definecolor{ThirdColor}{HTML}{F3890B}
\definecolor{BackgroundColor}{HTML}{E2E2E2}
\colorlet{name}{PrimaryColor}
\colorlet{tagline}{PrimaryColor}
\colorlet{heading}{PrimaryColor}
\colorlet{headingrule}{ThirdColor}
\colorlet{subheading}{SecondaryColor}
\colorlet{accent}{SecondaryColor}
\colorlet{emphasis}{SlateGrey}
\colorlet{body}{LightGrey}
%\pagecolor{BackgroundColor}   
% ----- DARK MODE -----
%\definecolor{BackgroundColor}{HTML}{242424}
%\definecolor{SlateGrey}{HTML}{6F6F6F}
%\definecolor{LightGrey}{HTML}{ABABAB}
%\definecolor{PrimaryColor}{HTML}{3F7FFF}
%\colorlet{name}{PrimaryColor}
%\colorlet{tagline}{PrimaryColor}
%\colorlet{heading}{PrimaryColor}
%\colorlet{headingrule}{PrimaryColor}
%\colorlet{subheading}{PrimaryColor}
%\colorlet{accent}{PrimaryColor}
%\colorlet{emphasis}{LightGrey}
%\colorlet{body}{LightGrey}
%\pagecolor{BackgroundColor}

% Change some fonts, if necessary
\renewcommand{\namefont}{\Huge\rmfamily\bfseries}
\renewcommand{\personalinfofont}{\small\bfseries}
\renewcommand{\cvsectionfont}{\LARGE\rmfamily\bfseries}
\renewcommand{\cvsubsectionfont}{\large\bfseries}

% Change the bullets for itemize and rating marker
% for \cvskill if you want to
\renewcommand{\itemmarker}{{\small\textbullet}}
\renewcommand{\ratingmarker}{\faCircle}

%% sample.bib contains your publications
%% \addbibresource{sample.bib}

\newcommand{\hrefc}[2]{\href{#1}{\textcolor{ThirdColor}{#2}}}

\begin{document}
    \name{Gaëtan Serré}
    \tagline{M2 Mathématiques, Vision, Apprentissage -- ENS Paris-Saclay}
    %% You can add multiple photos on the left or right
    \photoL{4cm}{gaetan_serre}
    
    \personalinfo{
        \email{gaetan.serre93@gmail.com}\smallskip
        \phone{+33 6 74 52 00 93}
        \location{Gif-Sur-Yvette, France}\\
        \github{gaetanserre}
        \homepage{gaetanserre.github.io/}
        %\medium{nicolasomar}
        %% You MUST add the academicons option to \documentclass, then compile with LuaLaTeX or XeLaTeX, if you want to use \orcid or other academicons commands.
        % \orcid{0000-0000-0000-0000}
        %% You can add your own arbtrary detail with
        %% \printinfo{symbol}{detail}[optional hyperlink prefix]
        % \printinfo{\faPaw}{Hey ho!}[https://example.com/]
        %% Or you can declare your own field with
        %% \NewInfoFiled{fieldname}{symbol}[optional hyperlink prefix] and use it:
        % \NewInfoField{gitlab}{\faGitlab}[https://gitlab.com/]
        % \gitlab{your_id}
    }
    
    \makecvheader
    %% Depending on your tastes, you may want to make fonts of itemize environments slightly smaller
    % \AtBeginEnvironment{itemize}{\small}
    
    %% Set the left/right column width ratio to 6:4.
    \columnratio{0.25}

    % Start a 2-column paracol. Both the left and right columns will automatically
    % break across pages if things get too long.
    \begin{paracol}{2}
        % ----- STRENGTHS -----
        \cvsection{Skills}
          \cvsubsection{Programming}
            \cvtag{Python} \cvtag{Tensorflow} \cvtag{Pytorch}
            \cvtag{SB3} \cvtag{RLlib}
            \cvtag{C++} \cvtag{CUDA} \cvtag{ONNX}
            \cvtag{OCaml}
            \cvtag{Java}
          \medskip
            
            \cvsubsection{Office software}
              \cvtag{\LaTeX}  
              \cvtag{Office suite}
        % ----- STRENGTHS -----

            
        
        % ----- LANGUAGES -----
        \cvsection{Languages}
            \cvlang{French}{Native}\\
            \divider
            
            \cvlang{English}{Fluent (TOEIC 945)}
            %% Yeah I didn't spend too much time making all the
            %% spacing consistent... sorry. Use \smallskip, \medskip,
            %% \bigskip, \vpsace etc to make ajustments.
            \smallskip
        % ----- LANGUAGES -----
        
        % ----- LOISIRS -----
        \cvsection{Hobbies}
            \cvlang{Chess} {\textasciitilde 1300 Elo}\\
            \divider
            \cvlang{Taekwondo} {Paris champion and IDF vice-champion in 2016}\\
            \divider\\
            \cvlang{Piano} {since 2020}\\
            \divider\\
            \cvlang{Escape Game} {since 2018}

            \smallskip
        % ----- LOISIRS -----
            
        
        % ----- MOST PROUD -----
        % \cvsection{Most Proud of}
        
        % \cvachievement{\faTrophy}{Fantastic Achievement}{and some details about it}\\
        % \divider
        % \cvachievement{\faHeartbeat}{Another achievement}{more details about it of course}\\
        % \divider
        % \cvachievement{\faHeartbeat}{Another achievement}{more details about it of course}
        % ----- MOST PROUD -----
        
        % \cvsection{A Day of My Life}
        
        % Adapted from @Jake's answer from http://tex.stackexchange.com/a/82729/226
        %\wheelchart{outer radius}{inner radius}{
        % comma-separated list of value/text width/color/detail}
        % \wheelchart{1.5cm}{0.5cm}{%
        %   6/8em/accent!30/{Sleep,\\beautiful sleep},
        %   3/8em/accent!40/Hopeful novelist by night,
        %   8/8em/accent!60/Daytime job,
        %   2/10em/accent/Sports and relaxation,
        %   5/6em/accent!20/Spending time with family
        % }
        
        % use ONLY \newpage if you want to force a page break for
        % ONLY the current column
        \newpage
        
        %% Switch to the right column. This will now automatically move to the second
        %% page if the content is too long.
        \switchcolumn
        
        % ----- ABOUT ME -----
        \cvsection{About me}
            \begin{quote}
              I am studying in the Artificial Intelligence master of Paris-Saclay.
              I am passionate about reinforcement learning, game theory,
              representation learning and computer graphics.
              I have done many projects, all available on my Github.
            \end{quote}
        % ----- ABOUT ME -----
        
        
        % ----- EDUCATION -----
        \cvsection{EDUCATION}
            \cvevent{M2 Mathématiques, Vision, Apprentissage (MVA) }{| ENS Paris-Saclay}
            {Sept. 2022 -- August 2023}{Gif-Sur-Yvette, France}
            \divider

            \cvevent{Master Artificial Intelligence }{| Université Paris-Saclay}{Sept. 2021 -- June 2022}{Gif-Sur-Yvette, France}
            Noteworthy courses:
            \vspace{4pt}
            \begin{itemize}
              \item Applied statistics (Supervisor: Marie-Anne Poursat)
              \item Mathematics for data science (Supervisor: Marcella Bonazzoli)
              \item Fundamental principles of machine learning (Supervisor: François Landes)
              \item Deep learning (Supervisor: Caio Corro)
            \end{itemize}
            \divider

            \cvevent{Double bachelor degree mathematics/computer science}{| Université Paris-Saclay}{Sept. 2018 -- June 2021}{Gif-Sur-Yvette, France}
            \begin{itemize}
                \item With honors
            \end{itemize}
            
        % ----- EDUCATION -----

        % ----- EXPERIENCE -----
        \cvsection{Experience}
        
        \cvevent{Internship | }{INRIA \& RTE}{May -- August 2022}{Gif-Sur-Yvette, France}
        \textbf{Supervisor:} Isabelle Guyon\\
        \vspace{3pt}
        Organization of the new edition of
        \hrefc{https://l2rpn.chalearn.org/}{Learning to Run a Power Network},
        a challenge joining reinforcement learning and electrical network
        in partnership with Réseau de Transport d'Électricité.
        Creation of a baseline agent merging RL and expert rules to stimulate the participants.
        I submitted a \hrefc{https://arxiv.org/abs/2207.10330}{paper} to IEEE SSCI as principal
        author \& presented this edition at the IEEE WCCI conference.

        \divider

        \cvevent{Internship | }{LMF -- INRIA}{May -- June 2021}{Gif-Sur-Yvette, France}
        \textbf{Supervisors:} Jean-Christophe Filliâtre \& Andrei Paskevich\\
        \vspace{3pt}
        The purpose of this internship was to improve \hrefc{http://why3.lri.fr/}{Why3},
        a deductive program verification software.
        This allowed me to become familiar with the world of research in a
        large computer science laboratory.

        \pagebreak

        % ----- CHALLENGES -----
        \cvsection{Challenges}
          I have participated in several artificial intelligence challenges:
          \vspace{4pt}
          \begin{itemize}
            \item \hrefc{https://codalab.lisn.upsaclay.fr/competitions/5410}{L2RPN WCCI 2022}
            \item \hrefc{https://competitions.codalab.org/competitions/25427}{L2RPN NEURIPS 2020 - Adaptability Track}
            \item \hrefc{https://codalab.lisn.upsaclay.fr/competitions/573}{Aerial Image Recognition}
          \end{itemize}

        % ----- PROJECTS -----
        \cvsection{Some projects}
          \cvevent{L2RPN 2022 PPO Baseline |}{\cvrepo{\faGithub}{https://github.com/gaetanserre/L2RPN-2022_PPO-Baseline}}
          {2022}{}
          The code of the baseline agent provided in the 2022 edition of
          Learning to Run a Power Network.\\
          \vspace{4pt}
          \cvtag{RL}
          \cvtag{SB3}
          \cvtag{Pytorch}\\
          \vspace{4pt}
          \divider

            \cvevent{GAiA |}{\cvrepo{\faGithub}{https://github.com/gaetanserre/GAiA}}
            {2021}{}
            A chess program that uses a complex residual neural network specialized in image recognition.\\
            You can read the report about GAiA
            \hrefc{https://raw.githubusercontent.com/gaetanserre/GAiA/master/report/Performing\%20Regression\%20on\%20Complex\%20Data.pdf}
            {here}.\\
            \vspace{4pt}
            \cvtag{CNN}
            \cvtag{Game theory}
            \cvtag{Pytorch}
            \cvtag{C++}
            \cvtag{ONNX}\\
            \vspace{4pt}
            \divider

            \cvevent{AlphaZero (RLlib) |}{\cvrepo{\faGithub}{https://github.com/gaetanserre/ray}}
            {2022}{}
            An implementation of the AlphaZero algorithm in a fork of the RLlib library.\\
            \vspace{4pt}
            \cvtag{RL}
            \cvtag{Python}
            \cvtag{Pytorch}\\
            \vspace{4pt}
            \divider
            
            \cvevent{VAE |}{\cvrepo{\faGithub}{https://github.com/gaetanserre/Variational-Auto-Encoder}}
            {2022}{}
            An implementation of a Variational Auto Encoder.\\
            \vspace{4pt}
            \cvtag{Python}
            \cvtag{Pytorch}\\
            \vspace{4pt}
        % ----- PROJECTS -----
        
    \end{paracol}
\end{document}
