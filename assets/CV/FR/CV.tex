%%%%%%%%%%%%%%%%%
% This is an sample CV template created using altacv.cls
% (v1.3, 10 May 2020) written by LianTze Lim (liantze@gmail.com). Now compiles with pdfLaTeX, XeLaTeX and LuaLaTeX.
% This fork/modified version has been made by Nicolás Omar González Passerino (nicolas.passerino@gmail.com, 15 Oct 2020)
%
%% It may be distributed and/or modified under the
%% conditions of the LaTeX Project Public License, either version 1.3
%% of this license or (at your option) any later version.
%% The latest version of this license is in
%%    http://www.latex-project.org/lppl.txt
%% and version 1.3 or later is part of all distributions of LaTeX
%% version 2003/12/01 or later.
%%%%%%%%%%%%%%%%

%% If you need to pass whatever options to xcolor
\PassOptionsToPackage{dvipsnames}{xcolor}

%% If you are using \orcid or academicons
%% icons, make sure you have the academicons
%% option here, and compile with XeLaTeX
%% or LuaLaTeX.
% \documentclass[10pt,a4paper,academicons]{altacv}

%% Use the "normalphoto" option if you want a normal photo instead of cropped to a circle
% \documentclass[10pt,a4paper,normalphoto]{altacv}

\documentclass[10pt,a4paper,ragged2e,withhyper]{altacv}

%% AltaCV uses the fontawesome5 and academicons fonts
%% and packages.
%% See http://texdoc.net/pkg/fontawesome5 and http://texdoc.net/pkg/academicons for full list of symbols. You MUST compile with XeLaTeX or LuaLaTeX if you want to use academicons.

% Change the page layout if you need to
\geometry{left=1.2cm,right=1.2cm,top=1cm,bottom=1cm,columnsep=0.75cm}

% The paracol package lets you typeset columns of text in parallel
\usepackage{paracol}

% Change the font if you want to, depending on whether
% you're using pdflatex or xelatex/lualatex
\ifxetexorluatex
  % If using xelatex or lualatex:
  \setmainfont{Roboto Slab}
  \setsansfont{Lato}
  \renewcommand{\familydefault}{\sfdefault}
\else
  % If using pdflatex:
  \usepackage[rm]{roboto}
  \usepackage[defaultsans]{lato}
  % \usepackage{sourcesanspro}
  \renewcommand{\familydefault}{\sfdefault}
\fi

% ----- LIGHT MODE -----
\definecolor{SlateGrey}{HTML}{2E2E2E}
\definecolor{LightGrey}{HTML}{666666}
\definecolor{PrimaryColor}{HTML}{001F5A}
\definecolor{SecondaryColor}{HTML}{0039AC}
\definecolor{ThirdColor}{HTML}{F3890B}
\definecolor{BackgroundColor}{HTML}{E2E2E2}
\colorlet{name}{PrimaryColor}
\colorlet{tagline}{PrimaryColor}
\colorlet{heading}{PrimaryColor}
\colorlet{headingrule}{ThirdColor}
\colorlet{subheading}{SecondaryColor}
\colorlet{accent}{SecondaryColor}
\colorlet{emphasis}{SlateGrey}
\colorlet{body}{LightGrey}
%\pagecolor{BackgroundColor}   
% ----- DARK MODE -----
%\definecolor{BackgroundColor}{HTML}{242424}
%\definecolor{SlateGrey}{HTML}{6F6F6F}
%\definecolor{LightGrey}{HTML}{ABABAB}
%\definecolor{PrimaryColor}{HTML}{3F7FFF}
%\colorlet{name}{PrimaryColor}
%\colorlet{tagline}{PrimaryColor}
%\colorlet{heading}{PrimaryColor}
%\colorlet{headingrule}{PrimaryColor}
%\colorlet{subheading}{PrimaryColor}
%\colorlet{accent}{PrimaryColor}
%\colorlet{emphasis}{LightGrey}
%\colorlet{body}{LightGrey}
%\pagecolor{BackgroundColor}

% Change some fonts, if necessary
\renewcommand{\namefont}{\Huge\rmfamily\bfseries}
\renewcommand{\personalinfofont}{\small\bfseries}
\renewcommand{\cvsectionfont}{\LARGE\rmfamily\bfseries}
\renewcommand{\cvsubsectionfont}{\large\bfseries}

% Change the bullets for itemize and rating marker
% for \cvskill if you want to
\renewcommand{\itemmarker}{{\small\textbullet}}
\renewcommand{\ratingmarker}{\faCircle}

%% sample.bib contains your publications
%% \addbibresource{sample.bib}

\begin{document}
    \name{Gaëtan Serré}
    \tagline{Master Artificial Intelligence -- Paris-Saclay}
    %% You can add multiple photos on the left or right
    \photoL{4cm}{gaetan_serre}
    
    \personalinfo{
        \email{gaetan.serre93@gmail.com}\smallskip
        \phone{+33 6 74 52 00 93}
        \location{Gif-Sur-Yvette, France}\\
        \github{Plagiat01}
        \homepage{Plagiat01.github.io/}
        %\medium{nicolasomar}
        %% You MUST add the academicons option to \documentclass, then compile with LuaLaTeX or XeLaTeX, if you want to use \orcid or other academicons commands.
        % \orcid{0000-0000-0000-0000}
        %% You can add your own arbtrary detail with
        %% \printinfo{symbol}{detail}[optional hyperlink prefix]
        % \printinfo{\faPaw}{Hey ho!}[https://example.com/]
        %% Or you can declare your own field with
        %% \NewInfoFiled{fieldname}{symbol}[optional hyperlink prefix] and use it:
        % \NewInfoField{gitlab}{\faGitlab}[https://gitlab.com/]
        % \gitlab{your_id}
    }
    
    \makecvheader
    %% Depending on your tastes, you may want to make fonts of itemize environments slightly smaller
    % \AtBeginEnvironment{itemize}{\small}
    
    %% Set the left/right column width ratio to 6:4.
    \columnratio{0.25}

    % Start a 2-column paracol. Both the left and right columns will automatically
    % break across pages if things get too long.
    \begin{paracol}{2}
        % ----- STRENGTHS -----
        \cvsection{Compétences}
            \cvsubsection{Programmation}
                \cvtag{Python} \cvtag{Tensorflow}\\
                \cvtag{Sklearn} \cvtag{Pandas} \cvtag{Numpy}
                \cvtag{Notebook}
                \cvtag{C++}
                \cvtag{CUDA}
                \cvtag{GLSL}
                \cvtag{Java}
                \medskip
            
            \cvsubsection{Bureautique}
                \cvtag{\LaTeX}
                \cvtag{Office Suite}
        % ----- STRENGTHS -----

            
        
        % ----- LANGUAGES -----
        \cvsection{Langues}
            \cvlang{Français}{Langue maternelle}\\
            \divider
            
            \cvlang{Anglais}{Courant / C1}
            %% Yeah I didn't spend too much time making all the
            %% spacing consistent... sorry. Use \smallskip, \medskip,
            %% \bigskip, \vpsace etc to make ajustments.
            \smallskip
        % ----- LANGUAGES -----
        
        % ----- LOISIRS -----
        \cvsection{Loisirs}
            \cvlang{Échecs} {\textasciitilde 1300 Elo}\\
            \divider
            \cvlang{Taekwondo} {champion Paris et vice-champion IDF en 2016}\\
            \divider\\
            \cvlang{Piano} {depuis 2020}\\
            \divider\\
            \cvlang{Escape Game} {depuis 2018}

            \smallskip
        % ----- LOISIRS -----
            
        
        % ----- MOST PROUD -----
        % \cvsection{Most Proud of}
        
        % \cvachievement{\faTrophy}{Fantastic Achievement}{and some details about it}\\
        % \divider
        % \cvachievement{\faHeartbeat}{Another achievement}{more details about it of course}\\
        % \divider
        % \cvachievement{\faHeartbeat}{Another achievement}{more details about it of course}
        % ----- MOST PROUD -----
        
        % \cvsection{A Day of My Life}
        
        % Adapted from @Jake's answer from http://tex.stackexchange.com/a/82729/226
        % \wheelchart{outer radius}{inner radius}{
        % comma-separated list of value/text width/color/detail}
        % \wheelchart{1.5cm}{0.5cm}{%
        %   6/8em/accent!30/{Sleep,\\beautiful sleep},
        %   3/8em/accent!40/Hopeful novelist by night,
        %   8/8em/accent!60/Daytime job,
        %   2/10em/accent/Sports and relaxation,
        %   5/6em/accent!20/Spending time with family
        % }
        
        % use ONLY \newpage if you want to force a page break for
        % ONLY the current column
        \newpage
        
        %% Switch to the right column. This will now automatically move to the second
        %% page if the content is too long.
        \switchcolumn
        
        % ----- ABOUT ME -----
        \cvsection{Profil}
            \begin{quote}
                J'étudie dans le master Artificial Intelligence de Paris-Saclay.
                Je suis passionné par l'intelligence artificielle, le machine learning, la théorie des jeux et
                la compilation.
                J'ai réalisé de nombreux projets, tous disponibles sur mon Github.
            \end{quote}
        % ----- ABOUT ME -----
        
        
        % ----- EDUCATION -----
        \cvsection{Formation}
            \cvevent{Master Artificial Intelligence }{| Université Paris-Saclay}{Sept. 2021 -- Juin 2023}{Orsay, France}
            Cours notables :
            \vspace{2pt}
            \begin{itemize}
                \item Applied statistics (Encadrant : Marie-Anne Poursat)
                \item Mathematics for data science (Encadrant : Marcella Bonazzoli)
                \item Fundamental principles of machine learning (Encadrant : François Landes)
                \item Deep learning (Encadrant : Caio Corro)
            \end{itemize}
            \divider

            \cvevent{Magistère Informatique }{| Université Paris-Saclay}{Sept. 2020 -- Juin 2021}{Orsay, France}
            \begin{itemize}
              \item Le but du magistère est d'avoir une initiation à la recherche à travers des stages, des cours supplémentaires et des conférences.
              \item 16.4/20
            \end{itemize}
            \divider

            \cvevent{Licence Double Diplôme Mathématiques/Informatique }{| Université Paris-Saclay}{Sept. 2018 -- Juin 2021}{Orsay, France}
            \begin{itemize}
                \item Mention Bien (15.16/20)
            \end{itemize}

            Cours notables :
            \vspace{2pt}
            \begin{itemize}
                \item Introduction à l'apprentissage statistique (3\textsuperscript{ème} année. Encadrant : François Landes)
                \item Compilation (3\textsuperscript{ème} année. Encadrant : Thibaut Balabonski)
                \item Projet ML: Projet en équipe de 6 ayant pour but de détecter des cellules infectées par la malaria à partir de photos
                      en utilisant des algorithmiques de machine learning.
                      (2\textsuperscript{ème} année. Encadrant : Isabelle Guyon)
            \end{itemize}
            
        % ----- EDUCATION -----

        \pagebreak

        % ----- PROJECTS -----
        \cvsection{Projets Personnels}
            \cvevent{GAiA |}{\cvrepo{\faGithub}{https://github.com/Plagiat01/GAiA}}
            {Avril 2021 -- Actuel}{}
            Un programme d'échecs qui utilise un réseau neuronal résiduel complexe spécialisé dans la reconnaissance d'images.\\
            Vous pouvez lire l'article que j'ai écrit à propos de GAiA
            \href{https://raw.githubusercontent.com/Plagiat01/GAiA/master/article/Performing%20Regression%20on%20Complex%20Data.pdf}{ici}\\ 
            \vspace{4pt}
            \cvtag{Tensorflow}
            \cvtag{Python}
            \cvtag{C++}\\
            \vspace{4pt}
            \divider

            \cvevent{Reinforcement learning |}{\cvrepo{\faGithub}{https://github.com/Plagiat01/Reinforcement-learning-examples}}
            {Nov. 2021 -- Actuel}{}
            Une implémentation d'un algorithme d'apprentissage par renforcement
            sur différents jeux (ex: les tours de Hanoï).\\
            \vspace{4pt}
            \cvtag{Notebook}
            \cvtag{Python}\\
            \vspace{4pt}
            \divider
            

            \cvevent{LiSA |}{\cvrepo{\faGithub}{https://github.com/Plagiat01/LiSA}}
            {Sept. 2020 -- Sept. 2021}{}
            Un moteur de rendu 3D en ray tracing.\\
            \vspace{4pt}
            \cvtag{C++}
            \cvtag{CUDA}\\
            \vspace{4pt}
        % ----- PROJECTS -----

        % ----- EXPERIENCE -----
            \cvsection{Experience Professionnelle}
            \cvevent{Stage | }{LMF \& INRIA}{Mai -- Juillet 2021}{Gif-Sur-Yvette, France}
            Ce stage avait pour but d'améliorer \href{http://why3.lri.fr/}{Why3}, un logiciel de vérification déductive de programme.
            Cela m'a permis de me familiariser avec le monde de la recherche au sein un grand laboratoire d'informatique.\\

            \divider

            \cvevent{Agent Commercial }{| SNCF}{Été 2018 -- 2021}{Paris, France}
            Vente et après-vente à la boutique grande ligne de Gare de l'Est. Cela m'a permis de me familiariser avec le monde de l'entreprise et le travail d'équipe.
        % ----- EXPERIENCE -----

    \end{paracol}
\end{document}
